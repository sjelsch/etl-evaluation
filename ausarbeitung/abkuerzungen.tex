%% abkuerzungen.tex
\markboth{Abk�rzungsverzeichnis}{Abk�rzungsverzeichnis}

\chapter*{Abk�rzungsverzeichnis}
\label{ch:Abkuerzungen}
%% ==============================

\begin{acronym}
\acro{API}{Application Programming Interface}
\acro{AWS}{Amazon Web Services}
\acro{BI}{Business Intelligence}
\acro{BIBM}{Business Intelligence Benchmark}
\acro{CDH}{Cloudera Distribution Including Apache Hadoop}
\acro{DSD}{Data Structure Definiton}
\acro{EC2}{Elastic Compute Cloud}
\acro{ETL-Prozess}{Extract-, Transform- und Load-Prozess}
\acro{FOAF}{Friend of a Friend}
\acro{HDFS}{Hadoop Distributed File System}
\acro{HiveQL}{Hive Query Language}
\acro{HTTP}{Hypertext Transfer Protocol}
\acro{JDBC}{Java Database Connection}
\acro{MDM}{Multidimensionales Datenmodell}
\acro{MDX}{Multidimensional Expression}
\acro{MOLAP}{Multidimensionales OLAP}
\acro{NoSQL}{Not only SQL}
\acro{OLAP}{On-Line Analytical Processing}
\acro{QB}{RDF Data Cube Vocabulary}
\acro{RDF}{Resource Description Framework}
\acro{RDFS}{RDF Schema}
\acro{ROLAP}{Relationales OLAP}
\acro{SPARQL}{SPARQL Protocol And RDF Query Language}
\acro{SQL}{Structured Query Language}
\acro{SSB}{Star Schema Benchmark}
\acro{URI}{Uniform Resource Identifier}
\acro{W3C}{World Wide Web Consortium}
\acro{XML}{Extensible Markup Language}
\end{acronym}

% \acro{Name}{Darstellung der Abk�rzung}{Langform der Abk�rzung}
% Folgendes benutzen, wenn der Plural einer Abk. ben�igt wird
% \newacroplural{Name}{Darstellung der Abk�rzung}{Langform der Abk�rzung}
% \newacroplural{Abk}[Abk-en]{Abk�rzungen}
