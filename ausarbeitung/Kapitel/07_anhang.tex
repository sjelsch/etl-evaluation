\begin{appendix}
\chapter{Anhang}
Im Rahmen der Abschlussarbeit wurde bei Github eine Projektseite angelegt. Diese ist unter der URL \url{https://github.com/sjelsch/etl-evaluation} aufrufbar. Auf dieser Seite ist der Quellcode des ETL-Prozesses sowie eine ausf�hrbare JAR-Datei mit Anleitung zu finden. Des Weiteren werden die Schritte f�r die Datengenerierung mit dem Star Schema Benchmark sowie die eingesetzte DataStructureDefinition-Datei auf der Seite beschrieben.

Neben einer Auflistung der verwendeten analytischen Abfragen beinhaltet diese Projektseite auch eine kurze Beschreibung der OLAP-Abfragen. Ferner sind die Ergebnisse der Evaluation f�r jeden einzelnen Durchgang aufgelistet.

\section{KylinDialect in Mondrian}
\label{anhang_kylin_dialect}

An dieser Stelle wird der entwickelte \textit{KylinDialect} f�r die Interaktion von Mondrian mit Apache Kylin vorgestellt. Dieser Quellcode wurde im Rahmen der Abschlussarbeit im Mondrian-Projekt in einem Pull Request\footnote{ s. Pull Request unter \url{https://github.com/pentaho/mondrian/pull/480}.} bei Github festgehalten.
\mbox{}\\
\begin{lstlisting}[language=Java,caption={Java-Klasse \textit{KylinDialect} in Mondrian.},label=list:kylin-sql-dialekt-complete]
package mondrian.spi.impl;

import java.sql.Connection;
import java.sql.SQLException;

/**
 * Implementation of {@link mondrian.spi.Dialect} for Kylin.
 *
 * @author S�bastien Jelsch
 * @since Dez 28, 2015
 */
public class KylinDialect extends JdbcDialectImpl {
    public static final JdbcDialectFactory FACTORY =
            new JdbcDialectFactory(KylinDialect.class, DatabaseProduct.KYLIN) {
                protected boolean acceptsConnection(Connection connection) {
                    return super.acceptsConnection(connection);
                }
            };

    /**
     * Creates a KylinDialect.
     */
    public KylinDialect(Connection connection) throws SQLException {
        super(connection);
    }

    @Override
    public boolean allowsCountDistinct() {
        return false;
    }

    @Override
    public boolean allowsJoinOn() {
        return true;
    }
}
\end{lstlisting}
\newpage
\section{RDF-Prefixe}
\label{sec:rdf-prefixe}
In Tabelle \ref{tab:verwendete-prefixe} werden alle in dieser Arbeit verwendeten Prefixe, die dazugeh�rigen URIs und die Namen der Vokabulare aufgelistet.

\begin{table}[h!]
\footnotesize
\centering
\begin{tabular}{l|l|l}
  \textbf{Prefix} & \textbf{Namespace URI} & \textbf{Vocabulary} \\ \hline \hline
  qb4o & \url{http://purl.org/qb4olap/cubes#} & Vocabulary for Business Intelligence \\ \hline
  qb & \url{http://purl.org/linked-data/cube#} & RDF Data Cube Vocabulary \\ \hline
  rdf & \url{http://www.w3.org/1999/02/22-rdf-syntax-ns#} & RDF Core \\ \hline
  rdfh-inst & \url{http://lod2.eu/schemas/rdfh-inst#} & BIBM RDF Vocabulary (Instances) \\ \hline
  rdfh & \url{http://lod2.eu/schemas/rdfh#} & BIBM RDF Vocabulary (Schema) \\ \hline
  rdfs & \url{http://www.w3.org/2000/01/rdf-schema#} & RDF Schema \\ \hline
  skos & \url{http://www.w3.org/2004/02/skos/core#} & Simple Knowledge Organization System \\ \hline
  skosclass & \url{http://ddialliance.org/ontologies/skosclass#} & SKOS extension for classifications \\ \hline
  xkos & \url{http://purl.org/linked-data/xkos#} & SKOS Extension for Statistics \\ \hline
  xsd & \url{http://www.w3.org/2001/XMLSchema#} & Schema Datatypes in RDF and OWL \\ \hline
\end{tabular}
\caption{Auflistung der verwendete Prefixe mit zugeh�rigen URIs und Vokabulare.}
\label{tab:verwendete-prefixe}
\end{table}

\end{appendix}
