\blankpage

\chapter*{Kurzzusammenfassung}
\addcontentsline{toc}{chapter}{Kurzzusammenfassung}

In den letzten Jahren ist die Menge der verf�gbaren Linked Data im World Wide Web stetig gestiegen. Immer mehr Provider ver�ffentlichen ihre statistischen Datens�tze nach dem Linked-Data-Prinzip, um diese Daten mit weiteren Informationen aus unterschiedlichen Quellen anreichern zu k�nnen. Durch die Verlinkung beliebiger Zusatzinformationen sollen die Daten n�her bestimmt und neue Erkenntnisse erlangt werden.

Bevor Analysten jedoch in der Lage sind, solche statistischen Daten vergleichen zu k�nnen, verbringen sie unverh�ltnism��ig viel Zeit mit der Identifizierung, Erfassung und Aufbereitung der relevanten Daten. Zus�tzlich f�hrt die zunehmende Gr��e an verf�gbaren RDF-Datens�tzen dazu, dass diese nicht mehr effizient auf einem einzelnen Rechner analysiert werden k�nnen. Aus diesem Grund sind f�r Analysen gro�er Datenmengen Technologien aus dem Big-Data-Umfeld notwendig, die diese Beschr�nkungen mittels Parallelisierung �ber viele Rechner hinweg �berwinden. In der Folge ist f�r die Analyse einer beliebig gro�en RDF-Datenmenge ein L�sungsansatz erforderlich, der diese Datens�tze generisch und automatisiert in ein horizontal skalierbares Open Source OLAP-System integriert, die Daten in geeigneter Form aufbereitet und dem Nutzer pr�sentiert.

In dieser Arbeit wird ein horizontal skalierbarer Extract-, Transform- und Load-Prozess (ETL-Prozess) konzipiert, umgesetzt und im Hinblick auf die Ausf�hrungszeit f�r die Integration der RDF-Daten in das OLAP-System Apache Kylin evaluiert. Des Weiteren werden die Antwortzeiten analytischer MDX- und SQL-Abfragen untersucht. Die Ergebnisse zeigen, dass der ETL-Prozess erst ab einer gr��eren Datenmenge einen Vorteil gegen�ber Import-Vorg�ngen in nicht-horizontal skalierenden Systemen bietet. Jedoch werden analytische Abfragen in diesem System aufgrund der horizontalen Skalierung in sehr kurzer Zeit beantwortet.

\textbf{Schl�sselw�rter:} Linked Data, Data Cube, Parallelisierung, MapReduce
